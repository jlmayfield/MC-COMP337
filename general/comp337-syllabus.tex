\documentclass[10pt]{article}
\usepackage{amsmath}
\usepackage{setspace}
\usepackage{hyperref}
\usepackage{booktabs}

\setlength{\textheight}{9in} \setlength{\topmargin}{-.5in}
\setlength{\textwidth}{6.5in} \setlength{\oddsidemargin}{0in}
\setlength{\evensidemargin}{0in}

\title{Syllabus \\ COMP 337 \\ Computer Communications}
\author{  }
\date{Spring 2024}

\begin{document}
\maketitle

\section{Logistics}
\begin{itemize}
\item \textbf{Where: }Center for Science and Business (CSB), Room 303
\item \textbf{When: } MWF 12--12:50pm
\item \textbf{Instructor: } Logan Mayfield
\begin{itemize}
\item \textit{Office: } Center for Science and Business (CSB), Room 344
\item \textit{Phone: } 309-457-2200 % chktex 8
\item \textit{Website: } \url{http://jlmayfield.github.io/}
\item \textit{Email: } lmayfield \textit{at} monmouthcollege \textit{dot} edu
\item \textit{Office Hours: }  MWF 9-10am. Tu 1-2pm. Th 10-11am. By appointment.
\end{itemize}
\item \textbf{Website: } \url{http://jlmayfield.github.io/teaching/COMP337/}
\item \textbf{Credits: } 1 course credit
\end{itemize}
\emph{Note: This Syllabus is subject to change based on specific class needs. Significant deviations from the syllabus will be discussed in class.}



\section{Description and Content}

In this course we examine the technology and fundamental principles governing communications between computers, i.e. computer networking. Careful attention will be paid to the modern Internet and the technologies that support it.  Students will use a combination of hands on network analysis and programming along with working problems from the book to explore the top four layers of the five protocol layers of computer networking: the application layer, transport layer, network layer, and the link layer. This study encompasses the first six chapters of the text. Time permitting, we'll examine basic issues in network security, the eighth chapter of the text.

\subsection{Textbook}

\noindent
Kurose, James F. \& Ross, Keith W. \textit{Computer Networking: A Top-Down Approach.} Pearson. 2021. ISBN:0-13-668155-7. % chktex 8


\section{Workload}

The weekly workload for this course will vary by student but on average should be about 13 hours per week.  The following table provides a rough estimate of the distribution of this time over different course components for a 16 week semester.

\begin{center}
\begin{tabular}{ll}
\underline{Assignment Type}  & \underline{Time/week} \\
Class   & 3 hours/week \\
Reading \& Review  & 2.5 hours/week \\
Problem Sets & 2 hours/week \\
Exam Study Time & 1 hours/week \\
Wireshark and Programming  & 2.25 hours/week \\
Portfolio Review & 0.25 \\
\bottomrule
 & 11 hours/week
\end{tabular}
\end{center}


The course assignment break down is as follows:

\begin{center}
  \begin{tabular}{ll}
    \underline{Category} & \underline{Number of Assignments} \\
    Problem Sets & 6 \\
    Wireshark Labs & 6-8 \\    
    Exams & 6 \\
    Programming Assignment & 1-3 \\
    Portfolio Review \& Self Evaluation Meetings & 4--5 \\
  \end{tabular}
\end{center}


\subsection*{WireShark Labs}

Every chapter will have a hands-on exercise using the Wireshark packet sniffer and other networking software. The results of these exercise will be discussed and dissected by the class during which time everyone is expected to participate and contribute.


\subsection*{Problem Sets}

Students will be assigned problems from each chapter. These problems will mirror the types of questions they can expect to be asked on the exam that will accompany each chapter. 

\subsection*{Exams}

Each chapter will conclude with an Exam. Expect these to be standard, in-class exams. Take-home exams are possible. 

\subsection*{Portfolio Review \& Self-Evaluation}

Self-reflection and self-evaluation is a critical component of learning and vital to a growth mindset.
We will keep a portfolio of the work you do throughout the semester. Much of this will be done automatically
by our assignment managment and version control software. At regular intervals throughout the semester you will
meet, one-on-one, with me to \textit{present your porfolio}, review items from your porfolio that best 
guage how well you're doing at meeting the course goals and expectations, and discuss how that success maps to 
a letter grade.  


\section{Ungrading \& Final Grades}

This class is largely ungraded. That means your assignments will not be graded for points and your final grade
is not determined by a point-based, numerical grading system. You will get feedback on your work but you will
see points on nothing. You don't earn points for doing work or getting something correct nor do you lose points
for getting something wrong. We're here to learn. Doing the work is how we do that and getting things wrong
some or most of the time is part of learning.

\subsection{Self-Evaluation \& Final Course Grades}

Throughout the semester you'll be asked to engage in regular self-evaluation. This process is described in
detail in additional documentation. Part of the process includes you self-assigning a course grade based on
your self-evaluation. Your self-evaluation and self-assigned grade are then discussed with me in a one-on-one
meeting during which we'll agree upon your current grade. The key here is that \textit{your self-evaluation
and self-assigned grade begins the conversation, not my assigned points.}

Below are some general rules of thumb we'll try to stick to when talking about grades. They relate grades to
course competency expectations and Monmouth College policy.
\begin{itemize}
  \item \textbf{A} - Exceeding course expectations.
  \item \textbf{B} - Meeting and occasionally exceeding course expectations.
  \item \textbf{C} - Meeting course expectations. \textit{This is the minimum grade required to continue on to COMP152. So, a C means you can be successful in a class that builds upon the things learned in this class.}
  \item \textbf{C-} - Mostly meeting course expectations. \textit{This is the minium grade that counts towards a major.}
  \item \textbf{D} - Occasionally meeting course expectations, but mostly not. \textit{Grades in the D range earn credit towards graduation but fall below GPA requirements.}
  \item \textbf{F} - Did not meet course expectations.
\end{itemize}

My hope is that the self-evaluation and self-directed grading process provides a lot of flexibility in terms
of how you can achieve success in this course and meet your grade goals. If you ever have questions or concerns
about self-evaluations and grades, then I'm more more than willing to discuss them with you at any time.

\subsubsection{Participation, Attendance, \& Timely Work}

I do not have strict attendance and deadline policies, per se, but I do have clear expectations. These
expectations are baked into the dispositional attribute of the course competencies. This attribute
includes things like being \textit{professional, responsible, responsive, and self-directed.}

As far as I'm concerned, signing up for this class means you agree to coming to class and lab,
being on time for class and lab, doing assigned work and submitting it on time, and generally participating
in all the class has to offer.  That being said, life happens and people have different priorities.
You might need to miss class or extend a deadline.  So long as you communicate with me about it, as a
professional would with a co-worker, then we won't have a problem. If you simply skip class without
warning, always show up late, or regularly fail to do assigned work in a timely manner, then I expect that
those failures to meet dispositional expectations to be reflected in your self-evaluation.

There is one exception to my ``no grade-based policy'' on assignments and deadlines and that is the
self-evaluations and reflections. The self-evaluation process is critical to this class and in no way
optional. \textbf{If you fail attend the portfolio review meetings or always show up completely un-prepared
then I reserve to give you a final grade of D or lower for the course.} You'll find I can be pretty relaxed
about a lot of other assignments and deadlines, but I draw the line at the self-evaluation process.

\subsection{Academic Honesty}

From the Monmouth College Academic Honesty Policy:
\begin{quote}
  ``We view academic dishonesty as a threat to the integrity and intellectual mission of our institution. Any breach of the academic honesty policy - either intentionally or unintentionally - will be taken seriously and may result not only in failure in the course, but in suspension or expulsion from the college. It is each student's responsibility to read, understand and comply with the general academic honesty policy at Monmouth College, as defined here in the Scots Guide, and to the specific guidelines for each course, as elaborated on the professor's syllabus.''

  ``The following areas are examples of violations of the academic honesty policy:
  \begin{enumerate}
  \item Cheating on tests, labs, etc;
  \item Plagiarism, i.e., using the words, ideas, writing, or work of another without giving appropriate credit;
  \item Improper collaboration between students, i.e., not doing one’s own work on outside assignments specified as group projects by the instructor;
  \item Submitting work previously submitted in another course, without previous authorization by the instructor.''
  \end{enumerate}

  ``Please note that this list is not intended to be exhaustive.''
\end{quote}

The complete Monmouth College Academic Honesty Policy can be found on the College web page by clicking on ``Student Life'' then on ``Scot’s Guide'' in the navigation bar to the left, then ``Academic Regulations'' in the navigation bar at the left.  Or you can visit the web page directly by typing in this URL: \url{https://ou.monmouthcollege.edu/life/residence-life/scots-guide/academic-regulations.aspx}

In this course, any violation of the academic honesty policy will have varying consequences depending on the severity of the infraction as judged by the instructor. Minimally, a violation will result in an``F'' or 0 points on the assignment in question. Additionally, the student’s course grade may be lowered by one letter grade. In severe cases, the student will be assigned a course grade of ``F'' and dismissed from the class. All cases of academic dishonesty will be reported to the Associate Dean who may decide to recommend further action to the Admissions and Academic Status Committee, including suspension or dismissal. It is assumed that students will educate themselves regarding what is considered to be academic dishonesty, so excuses or claims of ignorance will not mitigate the consequences of any violations.

\section{Accessibility}

Student Success \& Accessibility Services offers FREE resources to assist Monmouth College students with their academic success. Programs include Supplemental Instruction for select classes, Drop-In and appointment tutoring, and individual Academic Coaching. Our office is here to help all students excel academically, since all students can work toward better grades, practice stronger study skills, and manage their time better.

If you have a disability or had academic accommodations in high school or another college, you may be eligible for academic accommodations at Monmouth College under the Americans with Disabilities Act (ADA). Monmouth College is committed to equal educational access. To discuss any of the services offered, please call or meet with Robert Crawley, Interim Director of Student Success \& Accessibility Services.  SSAS is located in the new ACE space on the first floor of the Hewes Library, opposite Einstein’s Bros Bagels. They can be reached at 309-457-2257 or via email at: ssas@monmouthcollege.edu.

\section{Calendar}

\textit{This calendar is subject to change based on the circumstances of the course.} You'll find a more detailed, regularly updated calendar on the course website. This calendar covers the projected due dates of other assignments.


\begin{center}
  \begin{tabular}{lllll}
  \underline{Week} & \underline{Dates} & \underline{Notes} & \underline{Assignments Due} & \underline{Chapter(s)}\\
  1 & 1/17 --- 1/19 & &  & 1 \\
  2 & 1/22 --- 1/26 &  & & 1 \\
  3 & 1/29 --- 2/2 & & Hwk 1. Wireshark 1. & 1,2 \\
  4 & 2/5 --- 2/9 & & Exam 1 & 2 \\
  5 & 2/12 --- 2/16 & & Wireshark 2. Hwk 2. & 2 \\
  6 & 2/19 --- 2/23 & & Program 1. Exam 2. & 3 \\
  7 & 2/26 --- 3/1 & &  & 3 \\
  8 & 3/4 --- 3/8 & & Wireshark 3.  & 3 \\
  & 3/11 --- 3/15 &  SPRING BREAK & \\
  9 & 3/18 --- 3/22 & & Hwk 3. Wireshark 4.& 3 \\
  10 & 3/25 --- 3/29 & EASTER (F) & Exam 3.  & 4 \\
  11 & 4/1 --- 4/5 & EASTER (M) &  & 4 \\
  12 & 4/8 --- 4/12 & & Hwk 4. Wireshark 5. & 4 \\
  13 & 4/15 --- 4/19 & & Exam 4. & 5 \\
  14 & 4/22 --- 4/26 & SCHOLAR'S DAY (Tu). & Hwk5. Wireshark 6. & 5 \\
  15 & 4/29 --- 5/3 & &  Exam 5. & 6\\
  16 & 5/6 --- 5/10 & READING DAY (Th).  & \textbf{Exam 6. 5/10. 6:30-9:30 pm} & 6
  \end{tabular}
\end{center}
  
\end{document}
