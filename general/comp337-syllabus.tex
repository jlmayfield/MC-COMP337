\documentclass[10pt]{article}
\usepackage{amsmath}
\usepackage{setspace}
\usepackage{hyperref}
\usepackage{booktabs}

\setlength{\textheight}{9in} \setlength{\topmargin}{-.5in}
\setlength{\textwidth}{6.5in} \setlength{\oddsidemargin}{0in}
\setlength{\evensidemargin}{0in}

\title{Syllabus \\ COMP 337 \\ Computer Communications}
\author{  }
\date{Spring 2020}

\begin{document}
\maketitle

\section{Logistics}
\begin{itemize}
\item \textbf{Where: }Center for Science and Business (CSB), Room 303
\item \textbf{When: } MTWF 11--11:50am
\item \textbf{Instructor: } Logan Mayfield
\begin{itemize}
\item \textit{Office: } Center for Science and Business (CSB), Room 344
\item \textit{Phone: } 309-457-2200 % chktex 8
\item \textit{Website: } \url{http://jlmayfield.github.io/}
\item \textit{Email: } lmayfield \textit{at} monmouthcollege \textit{dot} edu
\item \textit{Office Hours: }  M 1:30--3pm, Tu 9--10am, Th 1--2pm, and by appointment.
\end{itemize}
\item \textbf{Website: } \url{http://jlmayfield.github.io/teaching/COMP337/}
\item \textbf{Credits: } 1 course credit
\end{itemize}
\emph{Note: This Syllabus is subject to change based on specific class needs. Significant deviations from the syllabus will be discussed in class.}



\section{Description and Content}

In this course we examine the technology and fundamental principles governing communications between computers, i.e. computer networking. Careful attention will be paid to the modern Internet and the technologies that support it.  Students will use a combination of hands on network analysis and programming along with working problems from the book to explore the top four layers of the five protocol layers of computer networking: the application layer, transport layer, network layer, and the link layer. This study encompasses the first six chapters of the text. Time permitting, we'll examine basic issues in network security, the eighth chapter of the text.

\subsection{Textbook}

\noindent
Kurose, James F. \& Ross, Keith W. \textit{Computer Networking: A Top-Down Approach.} Pearson. 2017. ISBN:0-13-359414-9. % chktex 8


\section{Workload}


The course workload is as follows:

\begin{center}
  \begin{tabular}{ll}
    \underline{Category} & \underline{Number of Assignments} \\
    Problem Sets & 6 \\
    Wireshark Labs & 6 \\
    Review Question Quizzes & 6 \\
    Exams & 6 \\
  \end{tabular}
\end{center}


\subsection*{WireShark Labs}

Every chapter will have a hands-on exercise using the Wireshark packet sniffer and other networking software. The results of these exercise will be discussed and dissected by the class during which time everyone is expected to participate and contribute.


\subsection*{Review Questions and Reading Quizzes}

This class will mostly run as a flipped classroom. Students are expected to do readings from the text prior to class and should be prepared to answer assigned \textit{review questions} for the readings. At the beginning of nearly every class period there will be a 4 to 5 question, multiple choice and true/false quiz based on the assigned reading and review questions. These quizzes will be administered using \textit{Socrative} and the results will tabulated on a chapter by chapter basis.  In effect, there is one quiz per chapter but the quiz is broken down into bite sized chunks and spread out over the course of a few weeks.

\subsection*{Problem Sets}

In addition to readings, review questions, and Wireshark labs, students will be assigned problems from each chapter. These problems will mirror the types of questions they can expect to be asked on the exam that will accompany each chapter except for chapters two and five. For these chapters students will carry out programming projects in place of an exam. These programs count towards their exam grade.

\subsection*{Exams}

At the completion of all but chapters two and five there will be an exam. For chapters two and five there will be a programming project in place of an exam. Each exam/program is weighted the same such that there will be no traditional midterm or final exam.

\subsection{Course Engagement Expectations}

The weekly workload for this course will vary by student but on average should be about 13 hours per week.  The following table provides a rough estimate of the distribution of this time over different course components for a 16 week semester.
\begin{center}
\begin{tabular}{ll}
\underline{Assignment Type}  & \underline{Time/week} \\
Class   & 4 hours/week \\
Reading \& Review  & 2 hours/week \\
Problem Sets & 2 hours/week \\
Exam Study Time & 1 hours/week \\
Projects  & 3 hours/week \\
\bottomrule
 & 12 hours/week
\end{tabular}
\end{center}


\section{Grades}

This course uses a standard grading scale.  Assignments and final grades will not be curved except in rare cases when its deemed necessary by the instructor.  Percentage grades translate to letter grades as follows:

\begin{center}
\begin{small}
\begin{tabular}{lcl}
Score & & Grade \\ \toprule
94--100 & & A \\
90--93 & & A- \\
88--89 & & B+ \\
82--87 & & B \\
80--81 & & B- \\
78--79 & & C+ \\
72--77 & & C \\
70--71 & & C- \\
68--69 & & D+ \\
62--67 & & D \\
60--61 & & D- \\
0--59 & & F
\end{tabular}
\end{small}
\end{center}


You are always welcome to challenge a grade that you feel is unfair or calculated incorrectly.  Mistakes made in your favor will never be corrected to lower your grade.  Mistakes made not in your favor will be corrected.  \textit{Basically, after the initial grading, your score can only go up as the result of a challenge.}

\subsection{Grade Weights}

Your final grade is based on a weighted average based on certain  assignment categories.  You should be able to estimate your current grade based on your scores and these weights.  You may always visit the instructor \textit{outside of class time} to discuss your current standing.

\begin{center}
  \begin{tabular}{ll}
  \underline{Category} & \underline{Weight} \\
    Exams & 36 \% \\ %6x6% each
    Problem Sets &  24 \% \\ %6x4 each
    Wireshark Labs & 24 \% \\ %6*4 each
    Reading Quizzes & 12 \% \\ %6*2 each
    Participation & 4\%
  \end{tabular}
\end{center}

\subsubsection*{Reading Quizzes and Participation}

Reading quizzes will begin at the start of class each day and conclude after not more than 5-8 minutes. Your score will be tallied for each chapter and the result will be curved such that an A for the chapter begins at 75\%. This keeps the quizzes low stress and leaves some room for absences. The pure participation portion of your grade will be based on how often you chime in to questions and problems during class.

\subsection{Attendance and Late Assignments}

These is not a strict attendance policy in this course but it should be clear that being regularly absent or late will have a negative impact on your reading quiz scores and overall course participation. That begin said, students with valid reasons for missing class will not be penalized for missed reading quizzes and lost opportunities to contribute to the class. \textit{Students should make every possible effort to inform the instructor of an absence before it happens.}

Assignment due dates for assignments are firm but extensions can and will be granted so long as they are negotiated ahead of time and the reason for them is sound.

\subsection{Academic Honesty}

From the Monmouth College Academic Honesty Policy:
\begin{quote}
  ``We view academic dishonesty as a threat to the integrity and intellectual mission of our institution. Any breach of the academic honesty policy - either intentionally or unintentionally - will be taken seriously and may result not only in failure in the course, but in suspension or expulsion from the college. It is each student’s responsibility to read, understand and comply with the general academic honesty policy at Monmouth College, as defined here in the Scots Guide, and to the specific guidelines for each course, as elaborated on the professor’s syllabus.''

  ``The following areas are examples of violations of the academic honesty policy:
  \begin{enumerate}
  \item Cheating on tests, labs, etc;
  \item Plagiarism, i.e., using the words, ideas, writing, or work of another without giving appropriate credit;
  \item Improper collaboration between students, i.e., not doing one’s own work on outside assignments specified as group projects by the instructor;
  \item Submitting work previously submitted in another course, without previous authorization by the instructor.''
  \end{enumerate}

  ``Please note that this list is not intended to be exhaustive.''
\end{quote}

The complete Monmouth College Academic Honesty Policy can be found on the College web page by clicking on ``Student Life'' then on ``Scot’s Guide'' in the navigation bar to the left, then ``Academic Regulations'' in the navigation bar at the left.  Or you can visit the web page directly by typing in this URL: \url{https://ou.monmouthcollege.edu/life/residence-life/scots-guide/academic-regulations.aspx}

In this course, any violation of the academic honesty policy will have varying consequences depending on the severity of the infraction as judged by the instructor. Minimally, a violation will result in an``F'' or 0 points on the assignment in question. Additionally, the student’s course grade may be lowered by one letter grade. In severe cases, the student will be assigned a course grade of ``F'' and dismissed from the class. All cases of academic dishonesty will be reported to the Associate Dean who may decide to recommend further action to the Admissions and Academic Status Committee, including suspension or dismissal. It is assumed that students will educate themselves regarding what is considered to be academic dishonesty, so excuses or claims of ignorance will not mitigate the consequences of any violations.

\section{Accessibility}

Student Success \& Accessibility Services offers FREE resources to assist Monmouth College students with their academic success. Programs include Supplemental Instruction for select classes, Drop-In and appointment tutoring, and individual Academic Coaching. Our office is here to help all students excel academically, since all students can work toward better grades, practice stronger study skills, and manage their time better.

If you have a disability or had academic accommodations in high school or another college, you may be eligible for academic accommodations at Monmouth College under the Americans with Disabilities Act (ADA). Monmouth College is committed to equal educational access. To discuss any of the services offered, please call or meet with Robert Crawley, Interim Director of Student Success \& Accessibility Services.  SSAS is located in the new ACE space on the first floor of the Hewes Library, opposite Einstein’s Bros Bagels. They can be reached at 309-457-2257 or via email at: ssas@monmouthcollege.edu.

\section{Calendar}

\textit{This calendar is subject to change based on the circumstances of the course.} You'll find a calendar of the required reading and review questions on the course website. This calendar covers the projected due dates of other assignments.

\begin{center}
\begin{tabular}{lll}
\underline{Week} & \underline{Dates} & \underline{Assignments Due}  \\
1 & 1/16 --- 1/17 &  \\
2 & 1/20 --- 1/24 & WireShark 1 (W). PSet 1(F). \\
3 & 1/27 --- 1/31 & PSet 1(M). Exam 1 (F)  \\
4 & 2/3 --- 2/7 & WireShark 2 (F).\\
5 & 2/10 --- 2/14 &  PSet 2 (M). Exam 2 (F). \\
6 & 2/17 --- 2/21 & WireShark 3 (F).  \\
7 & 2/24 --- 2/28 &  PSet 3 (M). Exam 3 (F).   \\
8 & 3/2 --- 3/5 & SPRING BREAK (F) \\
  & 3/9 --- 3/13 &  SPRING BREAK \\
9 & 3/16 --- 3/20 &   \\
10 & 3/23 --- 3/27 & PSet 4 (M). WireShark 4 (Tu).    \\
11 & 3/30 --- 4/3 &  Exam 4 (W).   \\
12 & 4/6 --- 4/10 &  EASTER (F) \\
13 & 4/13 --- 4/17 & EASTER (M). WireShark 5 (W). PSet 5 (F). \\
14 & 4/20 --- 4/24 & SCHOLAR'S DAY (Tu).     \\
15 & 4/27 --- 5/1 & Exam 5 (W).  \\
16 & 5/4 --- 5/8 & PSet 6 (M). WireShark 6. READING DAY (Th)  \\
Final's Week & 5/11 (6:30pm --- 9:30pm) & Exam 6.    \\
\end{tabular}
\end{center}

\end{document}
